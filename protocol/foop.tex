\documentclass[a4paper,10pt]{article}

\usepackage[usenames,dvipsnames]{color}
\usepackage{comment}
\usepackage[utf8]{inputenc}
\usepackage{listings}
\usepackage[pdfborder={0 0 0}]{hyperref}

\definecolor{OliveGreen}{cmyk}{0.64,0,0.95,0.40}
\definecolor{Gray}{gray}{0.5}

\lstset{
    language=Eiffel,
    basicstyle=\ttfamily,
    keywordstyle=\color{OliveGreen},
    commentstyle=\color{Gray},
    captionpos=b,
    breaklines=true,
    breakatwhitespace=false,
    showspaces=false,
    showtabs=false,
    numbers=left,
}

\title{VU Fortgeschrittene Objektorientierte Programmierung \\
       SS 2013}
\author{Jakob Gruber,
        Michael Osl,
        Patrick Kohlmayr,
        Martin Lackner}

\begin{document}

\maketitle

\section{Aufgabe 3: Eiffel}

% ------------------------------------------------------------------------------

\subsection{Frage 1}

\emph{Wie hoch ist der Aufwand in Eiffel, um Zusicherungen im Programmcode zu 
formulieren?}

\vspace{3mm}

Nachdem die Syntax sehr einfach gestaltet ist, ist der Aufwand vergleichbar mit der Formulierung von Zusicherungen innerhalb von Kommentaren. Man hat jedoch den Vorteil, dass die Zusicherungen vom Compiler und zur Laufzeit \"uberpr\"uft werden. Es ist aber zu erwarten, dass der initiale Aufwand durch eine robustere Applikation und weniger Debugging belohnt wird. Ein weiterer Vorteil ergibt sich aus der automatisierten Generierung von Dokumentation aus dem Code heraus.

Es gibt folgende Arten von Zusicherungen:

\begin{itemize}
    \item Preconditions: \verb+require+ Keyword
    \item Postconditions: \verb+ensure+ Keyword
    \item Invariants: \verb+invariant+
\end{itemize}

\begin{figure}
\begin{lstlisting}
class
    ASSERTIONS 

feature
    x : INTEGER
    lets_loop (val_from : INTEGER; val_to : INTEGER)
    require
        positive_from : val_from >= 0
        positive_to : val_to >= 0
        from_lower_to : val_from < val_to
    do
        from
            x := val_from
        invariant
            x <= val_to
        until
            x = val_to
        loop
            x := x + 1
        end

    ensure
        x = val_to
    end

    invariant
        positive_x : x > 0
end
\end{lstlisting}
\label{lst:assertprog}
\caption{Simple Klasse zur Demonstration von Zusicherungen}
\end{figure}


% ------------------------------------------------------------------------------

\subsection{Frage 2}

\emph{Wie stark wirkt sich die Überprüfung von Zusicherungen auf die Laufzeit aus?}

\vspace{3mm}

Die \"Uberpr\"ufung der Zusicherungen kann im Compiler beliebig eingestellt werden. M\"ogliche Optionen sind

\begin{itemize}
\item no: kein Laufzeiteffekt,
\item require: nur Preconditions,
\item ensure: zus\"atzlich Postconditions,
\item invariant: zus\"atzlich Invariants, und
\item all: Pre-, Postconditions, Invariants, Check instructions, loop (in)variants.
\end{itemize}

Man kann also je nach Bedarf die Zusicherungs\"uberpr\"ufung anpassen. Empfohlen werden
\emph{no} und \emph{require}, aber die beste Einstellung ist abh\"angig von dem am besten
geeigneten Trade-off zwischen Effizienz und Sicherheit.

Das Eiffel-Entwicklerhandbuch empfielt, Zusicherungen nur während der Entwicklungszeit zu aktivieren, da sie in erster Linie zum Aufdecken von Bugs gedacht sind\footnote{\url{http://docs.eiffel.com/book/method/et-design-contract-tm-assertions-and-exceptions\#Run-time_assertion_monitoring}}. 

Wir haben das Laufzeitverhalten von Zusicherungen anhand des Programmes aus Listing~\ref{lst:assertprog} getestet. Die Laufzeit hat sich bei einem Bereich von 1 bis 5000 bei 10000 Durchläufen nach Deaktivieren aller Zusicherungsprüfungen von 42.4 auf 26.5 Sekunden verringert. Ob man jedoch alle Zusicherungen oder nur vereinzelte überprüft, hat sich auf die Laufzeit nicht ausgewirkt.

% ------------------------------------------------------------------------------

\subsection{Frage 3}

\emph{Vorbedingungen dürfen im Untertyp nicht stärker und Nachbedingungen nicht 
schwächer werden um Ersetzbarkeit zu garantieren. Der Eiffel-Compiler überprüft 
diese Bedingungen. Ist es (trotz eingeschalteter Überprüfung von Zusicherungen) 
möglich diese Bedingungen zu umgehen? Wenn ja, wie?}

\vspace{3mm}


Prinzipell ist es nicht möglich, Vorbedingungen zu Verstärken und Nachbedingungen zu schwächen. Dies wird in Eiffel dadurch sichergestellt, dass man beim Überschreiben von Methoden in Untertypen gezwungen ist, das Keyword \verb+require else+ für Vor- und \verb+ensure then+ für Nachbedingungen zu verwenden, da ansonsten der Compiler eine Fehlermeldung ausgibt.

Das Keyword \verb+require else+ verknüpft die Vorbedingung des Übertyps mit der Vorbedingung des Untertyps anhand einer \textit{oder}-Bedingung.

% TODO Ich bin mir (fast) sicher, dass es auch eine Möglichkeit gibt, das zu Umgehen, aber das müssen wir noch abchecken.

% ------------------------------------------------------------------------------

\subsection{Frage 4}

\emph{Eiffel erlaubt kovariante Eingangsparametertypen. Unter welchen Bedingungen 
führt das zu Problemen, und wie äußern sich diese? Können Sie ein Programm 
schreiben, in dem die Verwendung kovarianter Eingangsparametertypen zu einer 
Exception führt?}

\vspace{3mm}

% TODO

\begin{comment}
See http://docs.eiffel.com/book/method/et-inheritance#Covariance.2C_anchored_declarations.2C_and_.22catcalls.22
\end{comment}

\begin{figure}
\begin{lstlisting}
local
    account: ACCOUNT
    person: PERSON
do
    create {STUDENT}person.make ("Joe")
    create {SENIOR_ACCOUNT}account.make (person)
    -- Exception
end
\end{lstlisting}
\caption{Runtime Exception ausgel\"ost durch einen catcall}
\end{figure}

\vspace{3mm}

% ------------------------------------------------------------------------------

\subsection{Frage 5}

\emph{Vereinfachen kovariante Eingangsparametertypen die Programmierung? Unter 
welchen Bedingungen ist das so?}

\vspace{3mm}

Es gibt Szenarien, in denen die Verwendung von kovarianten Eingangsparametertypen erwünscht sein kann, um Begebenheiten aus der tatsächlichen Umwelt zu modellieren. Dies wird in folgendem Beispiel verdeutlicht:

Ein \lstinline|Lkw-Lenker| ist ein Untertyp der Klasse \lstinline|Kfz-Lenker|. Die Klasse \lstinline|Kfz-Lenker| definiert eine Methode \lstinline|fahre(Fahrzeug)|, die als Eingangstype eine Klasse vom Typ \lstinline|Fahrzeug| erwartet. Der \lstinline|Lkw-Lenker| soll jedoch nur Fahzeuge vom Typ \lstinline|Lkw| lenken, welche ein Untertyp der Klasse \lstinline|Fahrzeug| ist. Definiert man nun eine Methode \lstinline|fahre(Lkw)| im Typ \lstinline|Lkw-Lenker|, dann verletzt man das Ersetzbarkeitsprinzip, weil ein \lstinline|Lkw-Lenker| nicht jeden beliebigen anderen \lstinline|Kfz-Lenker| ersetzen kann. Dennoch kann es in realistischen Umgebungsszenarien erforderlich sein, genau solche Umstände abzubilden.




%TODO

\begin{comment}
Von OOP: Mehrfach dynamisches Binden, Multimethoden.
\end{comment}

\end{document}
