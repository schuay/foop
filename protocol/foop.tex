\documentclass[a4paper,10pt]{article}

\usepackage[usenames,dvipsnames]{color}
\usepackage{comment}
\usepackage[utf8]{inputenc}
\usepackage{listings}

\definecolor{OliveGreen}{cmyk}{0.64,0,0.95,0.40}
\definecolor{Gray}{gray}{0.5}

\lstset{
    language=Java,
    basicstyle=\ttfamily,
    keywordstyle=\color{OliveGreen},
    commentstyle=\color{Gray},
    captionpos=b,
    breaklines=true,
    breakatwhitespace=false,
    showspaces=false,
    showtabs=false,
    numbers=left,
}

\title{VU Fortgeschrittene Objektorientierte Programmierung \\
       SS 2013}
\author{Jakob Gruber,
        Michael Osl,
        Patrick Kohlmayr,
        Martin Lackner}

\begin{document}

\maketitle

\section{Aufgabe 3: Eiffel}

% ------------------------------------------------------------------------------

\emph{Wie hoch ist der Aufwand in Eiffel, um Zusicherungen im Programmcode zu 
formulieren?}

\vspace{3mm}

% ------------------------------------------------------------------------------

\emph{Wie stark wirkt sich die Überprüfung von Zusicherungen auf die Laufzeit aus?}

\vspace{3mm}

Die \"Uberpr\"ufung kann zur Laufzeit beliebig eingestellt werden. M\"ogliche Optionen sind

\begin{itemize}
\item no: kein Laufzeiteffekt,
\item require: nur Preconditions,
\item ensure: zus\"atzlich Postconditions,
\item invariant: zus\"atzlich Invariants, und
\item all: Pre-, Postconditions, Invariants, Check instructions, loop (in)variants.
\end{itemize}

Man kann also je nach Bedarf die Zusicherungs\"uberpr\"ufung anpassen. Empfohlen werden
\emph{no} und \emph{require}, aber die beste Einstellung ist abh\"angig von dem am besten
geeigneten Trade-off zwischen Effizienz und Sicherheit.

\vspace{3mm}

% ------------------------------------------------------------------------------

\emph{Vorbedingungen dürfen im Untertyp nicht stärker und Nachbedingungen nicht 
schwächer werden um Ersetzbarkeit zu garantieren. Der Eiffel-Compiler überprüft 
diese Bedingungen. Ist es (trotz eingeschalteter Überprüfung von Zusicherungen) 
möglich diese Bedingungen zu umgehen? Wenn ja, wie?}

\vspace{3mm}

% ------------------------------------------------------------------------------

\emph{Eiffel erlaubt kovariante Eingangsparametertypen. Unter welchen Bedingungen 
führt das zu Problemen, und wie äußern sich diese? Können Sie ein Programm 
schreiben, in dem die Verwendung kovarianter Eingangsparametertypen zu einer 
Exception führt?}

\vspace{3mm}

% ------------------------------------------------------------------------------

\emph{Vereinfachen kovariante Eingangsparametertypen die Programmierung? Unter 
welchen Bedingungen ist das so?}

\vspace{3mm}

\begin{comment}
Von OOP: Mehrfach dynamisches Binden, Multimethoden.
\end{comment}


\end{document}
